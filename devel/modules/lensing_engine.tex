\documentclass[preprint]{aastex}

% packages for figures
\usepackage{graphicx}
% packages for symbols
\usepackage{latexsym,amssymb,hyperref}
% AMS-LaTeX package for e.g. subequations
\usepackage{amsmath}

%=====================================================================
% FRONT MATTER
%=====================================================================

\slugcomment{Draft \today}

%=====================================================================
% BEGIN DOCUMENT
%=====================================================================

\newcommand{\kmax}{\ensuremath{k_\mathrm{max}}}
\newcommand{\kmin}{\ensuremath{k_\mathrm{min}}}
\newcommand{\rmd}{\ensuremath{\mathrm{d}}}
\newcommand{\beq}{\begin{equation}}
\newcommand{\eeq}{\end{equation}}

\begin{document}

\title{Lensing engine testing and normalization convention (Issue \#248)}

\begin{abstract}
This document contains Rachel's notes on tests of the lensing engine
for GalSim Issue \#248.  There are a number of questions we need to
check to make sure that that software is doing what we want it to do,
and all the relevant tests and equations are described below.
\end{abstract}

\section{Introduction}

There are a few issues that we wish to check regarding the lensing
engine.  They are:

\begin{enumerate}
\item $k$ definitions: the lensing engine seems to omit the $2\pi$
  factor that seems standard, i.e., normally we say for some length
  scale $x$ the corresponding $k$ should be $2\pi/x$ but the lensing
  engine in its current form does $1/x$.
\item Overall normalization of shear variance: Currently the shear
  variance for a constant shear power $P(k)=P_0$ does not depend on
  the grid size/shape.  This seems non-standard, so we should
  investigate and understand this further.
\item We need to check the scaling of either the observed correlation
  function or power spectrum with $k$, to make sure it is done right.
\item We must check whether unit conversions are done properly.
  GalSim works in terms of arcsec, but it's common to input numbers in
  terms of radians, so we should make sure that this works.
\item What is the effect of interpolating between grid points?  does
  the interpolant have the expected impact on the shear power?
\item If we put in just $E$ or just $B$ mode power with a flat
  spectrum, the variances of $\gamma_1^2$ and $\gamma_2^2$ differ; is
  this expected for a flat power spectrum or sign of a problem?
\item We should make sure that when we use both $E$ and $B$ mode
  power, that we get the expected result, and that results with only
  $E$ or only $B$ are also sane.
\end{enumerate}

\section{Theory}

The lensing engine requires a shear power spectrum, $P(k)$.  We are
working in the flat-sky limit, so when we see expressions in terms of
$\ell$ we can swap $\ell$ and $k$, and
\beq
P = \frac{k^2 C_{\ell}}{2\pi}\equiv \frac{k^2 \Delta^2(k)}{2\pi}.
\eeq

If we identify pairs of galaxies and get the shears in a coordinate
system defined along the vector connecting them ($\gamma_+$) and at 45
degrees with respect to it ($\gamma_\times$), then we can compute
correlation functions of the $\gamma_+$ and $\gamma_\times$ values,
which we will call $\xi_{++}$ and $\xi_{\times\times}$.  Then the
standard cosmological correlation functions $\xi_{\pm}$ are defined as
\begin{align}
\xi_{\pm}(\theta) &=  \xi_{++}\pm \xi_{xx} \\
 &= \frac{1}{2\pi}\int k\,\rmd k \Delta^2(k) J_{0/4}(k\theta).
\end{align}

Since correlation functions are dimensionless, we immediately see that
$\Delta^2$ has dimensions of angle$^2$ and $P$ is dimensionless.

The variance of the shear values, 
\beq
\mathrm{Var}(\gamma) = \langle g_1^2 + g_2^2\rangle,
\eeq
is essentially $\xi_+(\theta=0)$.   Note that this is what we get for
the defined $\xi_+$ in the limit of $\theta$ going to zero, but that
equation was defined in terms of $\gamma_+$ and $\gamma_\times$ rather
than $\gamma_1$ and $\gamma_2$.  On a grid, it's not clear that we
can enforce/check behavior of Var($\gamma_1$) or Var($\gamma_2)$,
particularly if the corners are important (which will be the case if
there is a lot of shear power at small $k$), probably we should 
 only expect normal behavior for Var($\gamma$), but it's still worth
 verifying this.  So, combining several equations,
\beq\label{E:shearvar}
\mathrm{Var}(\gamma) = \frac{1}{2\pi}\int k\,\rmd k \Delta^2(k).
\eeq

None of these integrals have had limits on them.  Formally they should
go from the minimum to the maximum accessible $k$ on our grid.  If our
grid is defined by
\begin{align}
L &= \mbox{length of grid along one dimension (angular units)}\\
d &= \mbox{spacing between grid points (angular units)}\\
N &= \mbox{number of grid points along one dimension} = L/d
\end{align}

Here I have simply written $\Delta^2(k)$ but in principle there can be
two such functions, $\Delta^2_{E}$ and $\Delta^2_B$.  I believe these
should simply be summed in the above equations, but should check this.

\section{Comparison software}

We will compare against a completely independent piece of software,
Chris Hirata's spherical harmonic transform code which is described in
multiple papers (for example,
\href{http://adsabs.harvard.edu/abs/2004PhRvD..70j3501H}{Hirata
  et~al. 2004}).  This does not use the flat-sky
approach, but that should not be huge important of a difference even
for our $L=10$ deg.  It is something to bear in mind if we start
looking for agreement at a few \% level on the largest $\theta$ or
smallest $k$.  This software wants $C_\ell(\ell)$ as its inputs.

\section{Test of constant shear power}

Our first test uses constant shear power $P(k)=P_0$, or
$\Delta^2(k)=2\pi P_0/k^2$.  Putting this into Eq.~\ref{E:shearvar},
we expect a shear variance of
\begin{align}
\mathrm{Var}(\gamma) &= P_0 \int \frac{\rmd k}{k} \\
 &= P_0 \ln{\left(\frac{\kmax}{\kmin}\right)}
\end{align}

Note that since this depends on a ratio of $k$'s, it doesn't matter
what units we use for defining our grids, only on the number of grid
points.  We have, for a grid\footnote{Note that I'm using what I think
  the $k$ definition should be, i.e., with a $2\pi$.  That's not
  what's in the code right now.},
\begin{align}
\kmax &= \frac{2\pi}{d} \\
\kmin &= \frac{2\pi}{\sqrt{2}L}\\
\frac{\kmax}{\kmin} &= \sqrt{2}N.
\end{align}

So we expect a shear variance defined over both components that is
\beq
\mathrm{Var}(\gamma) = \ln{(\sqrt{2}N)} P_0.
\eeq

Chris's SHT code seems to give exactly this result, as tested using
two grid sizes.  In contrast, GalSim just gives $P_0$ (also tested
using two grid sizes).  This does not quite seem to make sense, but
let's move on for now.

\section{Test of non-constant shear power}

Constant shear power is rather unrealistic and leads to a lot of
emphasis on the large scales corresponding to box diagonals.  Let's
instead try this with $P(k) = 2 P_2 k^2$, or $\Delta^2(k) =
4 \pi P_2$.  This should have a few advantages: first, we can check
for differences in $\gamma_1^2$ vs. $\gamma_2^2$ in a more realistic
scenario.  Second, the shear variance (as I will show) explicitly
depends on the normalization of $k$, so we can use it for tests of
normalization convention and dependence on the $2\pi$ factor that I
think is missing from our $k$'s. Again going back to Eq.~\ref{E:shearvar},
we expect a shear variance of
\begin{align}
\mathrm{Var}(\gamma) &= 2 P_2 \int k\,\rmd k\\
 &= P_2 (\kmax^2-\kmin^2).
\end{align}
(Note regarding units: Since we want $P$ and shear variance to be dimensionless, $P_2$
has dimensions of angle$^2$.)

I start with Chris's code, which wants $\ell$ and $C_\ell$ in units of
radians, and use $P_2=0.01$ radians$^2$ for $E$ mode (no $B$ mode power).  When
using our default grid that is $10\times 10$ deg with $N=100$ per
dimension, then in terms of radians, we have $\kmin\sim 25.46$ and
$\kmax=3600$, so we expect shear variance (over both components) of
$1.3\times 10^5$.  For a grid that covers $5\times 5$ deg with $N=50$,
then $\kmin$ doubles but $\kmax$ is the same, so the shear variance
should be nearly identical (to sub-percent level).  For a third grid
that covers the same area as original but with $N=50$, $\kmin$ is same
as for the first grid, but $\kmax$ is halved, so the shear variance
should decrease by a factor of $\sim 4$.  Result: in all {\em three}
cases Chris's code gives a shear variance of $1.3\times 10^5$!  This
is reassuring in the first two cases, completely perplexing for the
last one.

For GalSim, I want to start {\em without} testing our unit conversion,
so I am going to convert to arcsec in advance.  We had $P_2=0.01$
radian$^2$, so I will first convert it to
$P_2=0.01 (3600(180/\pi))^2=4.2545\times 10^8$ arcsec$^2$ and $P(k)=2
P_2 k^2 = (8.51\times 10^{8}) k^2$ before trying this in
GalSim.  For GalSim I will also try the calculation for the three
types of grids:
\begin{verbatim}
import galsim
import numpy as np
test_ps = galsim.PowerSpectrum(lambda k : (8.51e8)*(k**2))
g1, g2 = test_ps.buildGriddedShears(grid_spacing=360., ngrid=100)
print np.var(g1), np.var(g2), np.var(g1)+np.var(g2)
g1, g2 = test_ps.buildGriddedShears(grid_spacing=360., ngrid=50)
print np.var(g1), np.var(g2), np.var(g1)+np.var(g2)
g1, g2 = test_ps.buildGriddedShears(grid_spacing=720., ngrid=50)
print np.var(g1), np.var(g2), np.var(g1)+np.var(g2)
\end{verbatim}
The result is that the variances (over both components) are around
$1070$, $1070$, $275$.  Good news: the relative variances behave
approximately as expected for the three grides.  Bad news: the
normalization is way off, by a factor of $0.0083$ for the default grid.

Sanity check: say we have some length scale like 3 degrees, which is
within the range of interesting scales.  This is 0.0524 radians, or
$k\sim120$ radian$^{-1}$.  So
then the power corresponding to those scales should be, according to
our original equation that we put into Chris's code, $P\sim
0.02(120^2)\sim 288$.  In contrast, we told GalSim things in arcsec.
3 degrees is 10800 arcsec, or $k\sim 5.8\times 10^{-4}$ arcsec$^{-1}$.
We told Galsim $P\sim (8.51\times 10^8)\times (5.8\times
10^{-4})^2\sim 288$.  Therefore I think I really did handle the
``manual'' unit conversion right.

We now have two mysteries in this test:
\begin{itemize}
\item Why does Chris's SHT code {\em not} give a shear variance that
  behaved the way I thought it should when I changed the grid.  His
  code's behavior is otherwise quite sane and able to be calculated
  easily (and, it did behave appropriately when i changed the grid for
  the flat PS case).
\item Our code at least gives the right behavior when I change the
  grid (unlike in the flat PS case!) but I don't understand the
  overall shear variance normalization.  I guess I shouldn't be too
  surprised by this, given that I didn't understand the normalization
  for the flat PS case.
\end{itemize}

\textbf{Didn't try the stuff that comes after this (yet):}

Then, try it with the version that takes everything in radians, to
check our units conversion.

Finally, try the version that takes $\Delta^2$.

\end{document}
