\documentclass[preprint,10pt]{aastex}

% packages for figures
\usepackage{graphicx}
% packages for symbols
\usepackage{latexsym,amssymb}
% AMS-LaTeX package for e.g. subequations
\usepackage{amsmath}

%=====================================================================
% FRONT MATTER
%=====================================================================

\slugcomment{Draft \today}

%=====================================================================
% BEGIN DOCUMENT
%=====================================================================

\newcommand{\gsobject}{{\tt GSObject}}
\newcommand{\image}{{\tt Image}}

\begin{document}

\title{GalSim Library Quick Reference}

\section{Overview}
The GalSim Library provides a number of Python classes and methods for
simulating astronomical images.  The fundamental work flow will
normally be something like:
\begin{itemize}

\item Construct a representation of your desired astronomical object
  as a single GalSim \gsobject~instance or in combination using the special {\tt Add} and
  {\tt Convolve} compound-type \gsobject s --- see Section \ref{sect:gsobjects}.
\item \emph{Optional}: Apply transformations such as shear or magnification using
  the methods of the resulting \gsobject~instance --- see Section \ref{sect:gsobjects}.
\item Draw the object into a GalSim \image~object representing a
  postage stamp image of your astronomical object.  This can be done
  using the {\tt draw()} or {\tt drawShoot()} methods carried by all
  \gsobject s for rendering images ({\tt drawShoot} uses photon
  shooting) --- see Section \ref{sect:gsobjects}.
\item \emph{Optional}: Add noise to the \image~using one of the GalSim random
  deviate classes --- see Section \ref{sect:random}.
\item \emph{Optional}: Add the postage stamp \image~to a subsection of
  a larger \image~ instance, or to a larger
  structure containing multiple \image~instances each derived from \gsobject s
  as described above --- see Section \ref{sect:image}.
\item Save the \image (s) to file in FITS (Flexible Image Transport
  System) format --- see Section \ref{sect:image}.
\end{itemize}

There are many examples of this workflow in the directory {\tt
  GalSim/examples/}, in the example scripts {\tt BasicDemo.py},
{\tt MultiObjectDemo.py} and {\tt RealDemo.py}.  We now provide a
brief, reference description of the GalSim classes and methods which
can be used in this workflow.

\newpage 

\section{GSObject classes and methods}\label{sect:gsobjects}
\subsection{Lists of available GSObjects}
\subsubsection*{Simple GSObjects}
In the order in which the classes
appear in {\tt GalSim/galsim/base.py}:
\begin{itemize}
\item[] Gaussian
\item[] Moffat
\item[] AtmosphericPSF
\item[] Airy
\item[] Kolmogorov
\item[] OpticalPSF
\item[] Pixel
\item[] Sersic
\item[] Exponential
\item[] DeVaucouleurs
\item[] RealGalaxy
\end{itemize}

\subsubsection*{Compound GSObjects}
In the order in which the classes
appear in {\tt GalSim/galsim/base.py}:
\begin{itemize}
\item[] Add
\item[] Convolve
\end{itemize}

\subsection{Initializing GSObjects}
All \gsobject s are initialized using a similar pattern, and

\begin{itemize}
\item {\tt Gaussian} \\
  Requires a single (i.e.\ at least one, but no more than one) 
\end{itemize}

\section{Random deviate classes and methods}\label{sect:random}


\section{Image classes and methods}\label{sect:image}


\end{document}
