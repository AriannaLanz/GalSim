\documentclass[preprint,10pt]{aastex}

% packages for figures
\usepackage{graphicx}
% packages for symbols
\usepackage{latexsym,amssymb,hyperref}
% AMS-LaTeX package for e.g. subequations
\usepackage{amsmath}

%=====================================================================
% FRONT MATTER
%=====================================================================

\slugcomment{Draft \today}

%=====================================================================
% BEGIN DOCUMENT
%=====================================================================

\newcommand{\gsobject}{{\tt GSObject}}
\newcommand{\image}{{\tt Image}}

\begin{document}

\title{GalSim Library Quick Reference}

\section{Overview}

\emph{BARNEY TODO: Tidy this whole thing up, make it look a lot less ugly,
maybe use an entirely different document class.}

The GalSim Library provides a number of Python classes and methods for
simulating astronomical images.  The fundamental work flow will
normally be something like:
\begin{itemize}

\item Construct a representation of your desired astronomical object
  as a single GalSim \gsobject~instance or in combination using the special {\tt Add} and
  {\tt Convolve} compound-type \gsobject s --- see Section \ref{sect:gsobjects}.
\item \emph{Optional}: Apply transformations such as shear or magnification using
  the methods of the resulting \gsobject~instance --- see Section \ref{sect:gsobjects}.
\item Draw the object into a GalSim \image~object representing a
  postage stamp image of your astronomical object.  This can be done
  using the {\tt draw()} or {\tt drawShoot()} methods carried by all
  \gsobject s for rendering images ({\tt drawShoot} uses photon
  shooting) --- see Section \ref{sect:gsobjects}.
\item \emph{Optional}: Add noise to the \image~using one of the GalSim random
  deviate classes --- see Section \ref{sect:random}.
\item \emph{Optional}: Add the postage stamp \image~to a subsection of
  a larger \image~ instance, or to a larger
  structure containing multiple \image~instances each derived from \gsobject s
  as described above --- see Section \ref{sect:image}.
\item Save the \image (s) to file in FITS (Flexible Image Transport
  System) format --- see Section \ref{sect:image}.
\end{itemize}

There are many examples of this workflow in the directory {\tt
  GalSim/examples/}, in the example scripts {\tt BasicDemo.py},
{\tt MultiObjectDemo.py} and {\tt RealDemo.py}.  We now provide a
brief, reference description of the GalSim classes and methods which
can be used in this workflow.

\newpage 

\section{GSObject classes and methods}\label{sect:gsobjects}
\subsection{Lists of available GSObjects}

There are currently 12 types of \gsobject. The first ten listed are
`simple' or `atomic' \gsobject s that can be initialized by providing
values for their required or optional parameters; the last two are
`compound' classes used to represent combinations of \gsobject s.  
In the order in which the classes appear in {\tt GalSim/galsim/base.py}:
\begin{itemize}
\item[$\circ$] \href{http://galsim-developers.github.com/GalSim/classgalsim_1_1base_1_1_gaussian.html}{\texttt{Gaussian}} --- a 2D Gaussian profile.
\item[$\circ$] \texttt{Moffat} --- a Moffat profile, used to approximate PSFs.
\item[$\circ$] \texttt{AtmosphericPSF} --- currently an image-based
  implementation of a Kolmogorov PSF (see below), but expected to evolve to
  use an image of a stochastically modelled atmospheric PSF in the near future.
\item[$\circ$] \texttt{Airy} --- an Airy PSF for ideal diffraction
  through a circular aperture, supports central obscuration.
\item[$\circ$] \texttt{Kolmogorov} --- the Kolmogorov PSF for long-exposure
  images through a turbulent atmosphere.
\item[$\circ$] \texttt{OpticalPSF} --- a simple model for non-ideal
  (aberrated) propagation through circular or square apertures with obscuration.
\item[$\circ$] \texttt{Pixel} --- used for integrating light onto square or
  rectangular pixels.
\item[$\circ$] \texttt{Sersic} --- the S\'{e}rsic family of galaxy light
  profiles.
\item[$\circ$] \texttt{Exponential} --- the Exponential disc, a S\'{e}rsic
  with index $n=1$.
\item[$\circ$] \texttt{DeVaucouleurs} --- commonly used to model galaxy bulge
  profiles, a S\'{e}rsic with index $n=4$.
\item[$\circ$] \texttt{RealGalaxy} --- models galaxies using real
  data, including correction for the original PSF.
\item[$\circ$] \texttt{Add} --- a \emph{compound} object used for
  summing multiple \gsobject s.
\item[$\circ$] \texttt{Convolve} --- a \emph{compound} object used for
convolving multiple \gsobject s
\end{itemize}

For more info and initialization details for each \gsobject, type {\tt print
  galsim.<GSObject\_name>.\_\_doc\_\_} within the Python interpreter.

\subsection{GSObject methods}


\section{Random deviate classes and methods}\label{sect:random}


\section{Image classes and methods}\label{sect:image}


\end{document}
